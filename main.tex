%!TEX program = xelatex
\documentclass[xetex]{beamer}
\usetheme{PaloAlto}
\usecolortheme{default}
% !TEX root = ../main.tex
\usepackage{hyperref}
\usepackage{float}
\usepackage{graphicx}		% for pdf, bitmapped graphics files
\graphicspath{ {figs/} }
\usepackage{mathtools} 		% assumes mathtools package installed
\usepackage{amssymb}  		% assumes amssymb package installed
\usepackage{amsthm}
\usepackage{caption}
\usepackage{subcaption}
\usepackage{setspace}
\usepackage{listings}
\usepackage{algorithm}
\usepackage{algpseudocode}
\usepackage{tikz}
\usetikzlibrary{positioning}
\usetikzlibrary{patterns}
\usepackage{xcolor}
\usepackage{tabularx,lipsum}
\usepackage{xltxtra}
\usepackage{xgreek}
\usepackage{minted}
\usepackage{chemfig}
\usepackage{fontspec}
\usepackage{booktabs}
\setmainfont[Mapping=tex-text]{GFS Didot}
\setsansfont[Mapping=tex-text]{GFS Didot}
\usepackage{xunicode}
\usepackage{verbatim}
\usepackage{showexpl}
\let\orilabel\label
\lstloadlanguages{[LaTeX]Tex}
\lstset{%
     basicstyle=\ttfamily\small,
     commentstyle=\itshape\ttfamily\small,
     showspaces=false,
     showstringspaces=false,
     breaklines=true,
     breakautoindent=true,
     captionpos=t,
     language=[LaTeX]TeX,
     keywordstyle=\color{blue},
     identifierstyle=\color{magenta},
     extendedchars=true,
     inputencoding=utf8,
     pos=r,
     backgroundcolor=\color{yellow},
     frame=single,
     preset=\let\label\orilabel,
}
\lstset{explpreset={numbers=none}}

\makeatletter
% This is the vertical rule that is inserted
\def\therule{\makebox[\algorithmicindent][l]{\hspace*{.5em}\vrule height .75\baselineskip depth .25\baselineskip}}%

\newtoks\therules% Contains rules
\therules={}% Start with empty token list
\def\appendto#1#2{\expandafter#1\expandafter{\the#1#2}}% Append to token list
\def\gobblefirst#1{% Remove (first) from token list
  #1\expandafter\expandafter\expandafter{\expandafter\@gobble\the#1}}%
\def\LState{\State\unskip\the\therules}% New line-state
\def\pushindent{\appendto\therules\therule}%
\def\popindent{\gobblefirst\therules}%
\def\printindent{\unskip\the\therules}%
\def\printandpush{\printindent\pushindent}%
\def\popandprint{\popindent\printindent}%

%      ***      DECLARED LOOPS      ***
% (from algpseudocode.sty)
\algdef{SE}[WHILE]{While}{EndWhile}[1]
  {\printandpush\algorithmicwhile\ #1\ \algorithmicdo}
  {\popandprint\algorithmicend\ \algorithmicwhile}%
\algdef{SE}[FOR]{For}{EndFor}[1]
  {\printandpush\algorithmicfor\ #1\ \algorithmicdo}
  {\popandprint\algorithmicend\ \algorithmicfor}%
\algdef{S}[FOR]{ForAll}[1]
  {\printindent\algorithmicforall\ #1\ \algorithmicdo}%
\algdef{SE}[LOOP]{Loop}{EndLoop}
  {\printandpush\algorithmicloop}
  {\popandprint\algorithmicend\ \algorithmicloop}%
\algdef{SE}[REPEAT]{Repeat}{Until}
  {\printandpush\algorithmicrepeat}[1]
  {\popandprint\algorithmicuntil\ #1}%
\algdef{SE}[IF]{If}{EndIf}[1]
  {\printandpush\algorithmicif\ #1\ \algorithmicthen}
  {\popandprint\algorithmicend\ \algorithmicif}%
\algdef{C}[IF]{IF}{ElsIf}[1]
  {\popandprint\pushindent\algorithmicelse\ \algorithmicif\ #1\ \algorithmicthen}%
\algdef{Ce}[ELSE]{IF}{Else}{EndIf}
  {\popandprint\pushindent\algorithmicelse}%
\algdef{SE}[PROCEDURE]{Procedure}{EndProcedure}[2]
   {\printandpush\algorithmicprocedure\ \textproc{#1}\ifthenelse{\equal{#2}{}}{}{(#2)}}%
   {\popandprint\algorithmicend\ \algorithmicprocedure}%
\algdef{SE}[FUNCTION]{Function}{EndFunction}[2]
   {\printandpush\algorithmicfunction\ \textproc{#1}\ifthenelse{\equal{#2}{}}{}{(#2)}}%
   {\popandprint\algorithmicend\ \algorithmicfunction}%
\makeatother
% !TEX root = ../main.tex
% Some custom commands for the creation of the presentation
% \AtBeginSection[]
% {
%   \begin{frame}
%     \frametitle{Περιεχόμενα}
%     \tableofcontents[currentsection]
%   \end{frame}
% }
% Define a variable as a length
\newcommand{\nvar}[2]{%
\newlength{#1}
\setlength{#1}{#2}
}
% Define a few constants for drawing
\nvar{\dg}{0.3cm}
\def\dw{0.25}
\def\dh{0.5}
\def\mbw{2.5}
\def\mbl{5}
% Define commands for links, joints and such
\def\link{\draw [double distance=1.5mm, very thick] (0,0)--}
\def\joint{
  \filldraw [fill=white] (0,0) circle (5pt);
  \fill[black] circle (2pt);
}
\def\grip{
  \draw[ultra thick](0cm,5mm)--(0cm,-5mm);
  \fill (0cm, 2.5mm)+(0cm,2.5mm) -- +(1.6*5mm,0cm) -- +(0pt,-2.5mm);
  \fill (0cm, -2.5mm)+(0cm,2.5mm) -- +(1.6*5mm,0cm) -- +(0pt,-2.5mm) node[color=blue,anchor=north west]{$\{T\}$};
  \draw[thick,color=blue,->] (0,0) -- (1,0);
  \draw[thick,color=blue,->] (0,0) -- (0,1);
}
\def\robotbase{
  \fill[white] (-10pt,-20pt) rectangle (10pt,20pt);
  \draw[ultra thick] (-10pt,-20pt) -- (-10pt,20pt) -- (10pt,20pt) -- (10pt,-20pt) -- (-10pt,-20pt);
  \draw[ultra thick] (-10pt,0) -- (10pt,0);
  \fill[pattern=north east lines] (-10mm,0mm) rectangle (10mm,-10mm);
}
\def\mobilebase{
  %base
  \draw[ultra thick] (-\mbl/2,-\mbw/2)--(\mbl/2,-\mbw/2)--(\mbl/2,\mbw/2)--(-\mbl/2,\mbw/2)--(-\mbl/2,-\mbw/2);
  %first wheel
  \draw[ultra thick] (-\mbl*0.375,-\mbw/2) -- (-0.125*\mbl,-\mbw/2) -- (-0.125*\mbl,-0.6*\mbw) -- (-\mbl*0.375,-0.6*\mbw) -- (-\mbl*0.375,-\mbw/2);
  \fill[black] (-\mbl*0.375,-\mbw/2) rectangle (-0.125*\mbl,-0.6*\mbw);
  %second wheel
  \draw[ultra thick] (\mbl*0.375,-\mbw/2) -- (0.125*\mbl,-\mbw/2) -- (0.125*\mbl,-0.6*\mbw) -- (\mbl*0.375,-0.6*\mbw) -- (\mbl*0.375,-\mbw/2);
  \fill[black] (\mbl*0.375,-\mbw/2) rectangle (0.125*\mbl,-0.6*\mbw);
  %third wheel
  \draw[ultra thick] (-\mbl*0.375,\mbw/2) -- (-0.125*\mbl,\mbw/2) -- (-0.125*\mbl,0.6*\mbw) -- (-\mbl*0.375,0.6*\mbw) -- (-\mbl*0.375,\mbw/2);
  \fill[black] (-\mbl*0.375,\mbw/2) rectangle (-0.125*\mbl,0.6*\mbw);
  %fourth wheel
  \draw[ultra thick] (\mbl*0.375,\mbw/2) -- (0.125*\mbl,\mbw/2) -- (0.125*\mbl,0.6*\mbw) -- (\mbl*0.375,0.6*\mbw) -- (\mbl*0.375,\mbw/2);
  \fill[black] (\mbl*0.375,\mbw/2) rectangle (0.125*\mbl,0.6*\mbw);
}
\tikzset{
    elli/.style args={#1:#2and#3}{
        draw,
        shape=ellipse,
        rotate=#1,
        minimum width=2*#2,
        minimum height=2*#3,
        outer sep=0pt,
    },
    /pgf/decoration/raise/.append code={
        \def\tikzdecorationsbrace{#1}
    },
    elli node/.style={
        circle,
        black,
        draw=none,
        midway,
        anchor=#1-90,
        inner sep=0pt,
        shift=(#1+90:\tikzdecorationsbrace+\pgfdecorationsegmentamplitude)
    },
    eigen/.style 2 args={
        decorate,
        decoration={
            brace,
            amplitude=#1,
            mirror,
            raise=#2,
        },
    },
    eigen/.default={15pt}{4pt},
    axis/.style={
        line width=.5,
        ->,
    },
    normal axis/.style={
        axis,
        dashed,
    }
}

\title{\LaTeX{} Workshop}
\author{Άρης Συνοδινός}
\institute{Πανεπιστήμιο Πατρών}
\date{\today}

\begin{document}
\begin{frame}[plain]
\titlepage
\end{frame}

\section{Εισαγωγικά}

\begin{frame}{Το \TeX{}}
	\begin{block}{\TeX{}}
		Σύστημα στοιχειοθεσίας που αναπτύχθηκε από τον Donald E. Knuth
		 (Stanford) για την δημιουργία άρτιων κειμένων, ειδικά επιστημονικών
		 με μαθηματικά. 
	\end{block}
	\begin{exampleblock}{\LaTeX{}}
		Επέκταση του \TeX{} που αναπτύχθηκε από τον Leslie Lamport, με
		στόχο την εύκολη στοιχειοθεσία κειμένου με χρήση έτοιμων macro-εντολών.
	\end{exampleblock}
	\begin{alertblock}{\XeTeX{}}
		Επέκταση του \LaTeX{} που αναπτύχθηκε από τον Jonathan Kew, που υποστηρίζει
		Unicode κείμενα, μοντέρνες γραμματοσειρές και πολλά άλλα.
	\end{alertblock}
\end{frame}

\begin{frame}{Πλεονεκτήματα}
	\begin{itemize}
		\item Το αποτέλεσμα μοιάζει με επαγγελματικού επιπέδου στοιχειοθεσία.
		Η αλλαγή του στυλ συνήθως είναι αλλαγή μίας γραμμής.
		\item Εύκολη χρήση μαθηματικών
		\item Δεν απαιτείται γνώση τυπογραφίας
		\item Πολύπλοκα επιστημονικά κείμενα διαχειρίζονται απλά
		\begin{itemize}
			\item Βιβλιογραφία
			\item Ευρετήριο
			\item Ετεροαναφορές
			\item Πίνακας περιεχομένων, λίστα γραφημάτων/πινάκων κλπ...
		\end{itemize}
		\item Ανεξάρτητο λειτουργικού συστήματος/αρχιτεκτονικής
		\item Ανοιχτού κώδικα
		\item Κείμενο ASCII/Unicode (VCS/Storage κλπ)
	\end{itemize}
\end{frame}

\begin{frame}{Μειονεκτήματα}
	\begin{itemize}
		\item Δύσκολη πρώτη επαφή (απότομη καμπύλη εκμάθησης)
		\item Δύσκολος σχεδιασμός νέου στυλ
		\item Τα πακέτα είναι μεγάλα σε απαιτήσεις χώρου
		\item Πάρα πολλές αλληλεπικαλύψεις πακέτων
	\end{itemize}
\end{frame}

\subsection{Δομή}

\begin{frame}[containsverbatim]{Δομή ενός αρχείου}
\begin{lstlisting}
\documentclass[options]{class}
\usepackage{name1,name2,name3}
\usepackage[options]{name4}
\author{Name Surname}
\title{Title}
\date{\today}
\begin{document}
...
\end{document}
\end{lstlisting}
\end{frame}

\section{Προοίμιο}
\begin{frame}[containsverbatim]{Κλάσεις}
	\begin{tabular}{c|l}
		article & Για άρθρα γενικού τύπου \\
		minimal & Όσο πιο απλό γίνεται \\
		report  & Για αναφορές\\
		book    & Για βιβλία \\
		slides  & Για διαφάνειες \\
		memoir  & Για απομνημονεύματα \\
		letter  & Για γράμματα \\
		beamer  & Για παρουσιάσεις
	\end{tabular}
\end{frame}

\begin{frame}[containsverbatim]{Ρυθμίσεις}
	\begin{tabular}{c|l}
		font size	& 10pt | 11pt | 12pt \ldots\\
		paper size	& a4paper | legalpaper \ldots\\
		equation	& fleqn | leqno\\
		title 		& titlepage | notitlepage\\
		columns		& onecolumn | twocolumn\\
		printing	& oneside | twoside
	\end{tabular}
\end{frame}

\begin{frame}[containsverbatim]{Ειδικοί χαρακτήρες}
\begin{tabular}{lll}
    $\backslash$~~   & Έναρξη εντολής       	& {\color{green}\verb+$\backslash$+}\\
                     & σημείωση:              	& {\color{green}\verb+\\+} = newline\\
    \$               & Λειτουργία Μαθηματικών 	& {\color{green}\verb+\$+}          \\
    \&               & Πινακοποιητής			& {\color{green}\verb+\&+}          \\
    \%               & Σχόλιο					& {\color{green}\verb+\%+}          \\
    \#               & Χρησιμοποιείται στις εντολές	& {\color{green}\verb+\#+} 		\\
    \textasciitilde  &                      & {\color{green}\verb+\textasciitilde+} \\
    \textbar         & Κάθετες γραμμές σε πίνακες & {\color{green}\verb+\textbar+}  \\
    \_               & Δείκτης			    & {\color{green}\verb+\_+}              \\
    \textasciicircum & Εκθέτης    			& {\color{green}\verb+\textasciicircum+}\\
    \{ \}            & Κλείσιμο εντολής    	& {\color{green}\verb+\{ \}+}           \\
    $[\ ]$           & Παράμετροι εντολής   & {\color{green}\verb+$[ ]$+}           \\
    `` ''            & Εισαγωγικά	      	& {\color{green}\verb+`` ''+}           \\
    $> <$            & Πριν και μετά από στήλη & {\color{green}\verb+$> <$+}           \\
  \end{tabular}
\end{frame}

\begin{frame}[containsverbatim]{Πακέτα και ρυθμίσεις}
\begin{lstlisting}
\usepackage{package name}
\usetheme{theme name}
\setmainfont[options]{font}
\setsansfont[options]{font}
\newcommand{name}[num]{definition}
\newenvironment{name}[num]{before}{after}
% Relative Path
\graphicspath{ {folder/} }
% Absolute Path
\graphicspath{ {/path/to/folder/} }
\end{lstlisting}
\end{frame}

\section{Έγγραφο}
\begin{frame}[containsverbatim]{Βασική Δομή}
\begin{lstlisting}
\part{}
\chapter{}
\section{}
\subsection{}
\subsubsection{}
\paragraph{}
\subparagraph{}
\end{lstlisting}
\end{frame}

\begin{frame}[containsverbatim]{Αρθρωτό Κείμενο}
\begin{lstlisting}
\input{}	% Reads file as it is
\include{}	% No nesting, Page breaks
\includeonly{} % Instructs what to include (preamble)
\end{lstlisting}
\end{frame}

\begin{frame}{Συνεργασία - Εργαλεία}
\begin{itemize}
  \item ShareLaTeX
  \item Version Control Systems
    \begin{itemize}
      \item Απλό κείμενο - Diff!
      \item \bf{Νέα γραμμή σε κάθε πρόταση!}
    \end{itemize}
  \item Κοινή βιβλιογραφία
\end{itemize}
\begin{block}{Εργαλεία}
Πολλά εργαλεία που ξέρετε χρησιμοποιούν ήδη \LaTeX{}!\\
Matlab \qquad Doxygen \qquad GeoGebra \qquad R \qquad GNUPlot \qquad Python \qquad \LuaTeX{}
\end{block}
\end{frame}

\subsection{Περιεχόμενα}

\begin{frame}[containsverbatim]{Περιεχόμενα και λοιπά}
\begin{lstlisting}
\tableofcontents
\listoffigures
\listoftables
...
\bibliographystyle{plain}
\bibliography{filename.bib}
\printindex
\end{lstlisting}
\end{frame}

\subsection{Μορφοποίηση}

\begin{frame}[containsverbatim]{Διαμόρφωση Κειμένου 1}
\begin{LTXexample}
\textit{italics}

\textsl{slanted}

\emph{emphasizing}

\textsc{small caps}

\textbf{bold face}

\textsf{sans serif}

\texttt{typewriter}
\end{LTXexample}
\end{frame}

\begin{frame}[containsverbatim]{Διαμόρφωση Κειμένου 2}
\begin{LTXexample}
\tiny{tiny size}
\scriptsize{script size}
\footnotesize{footnote size}
\small{small size}
\normalsize{normal size}
\large{large size}
\Large{Large size}
\LARGE{LARGE size}
\huge{huge size}
\Huge{Huge size}
\end{LTXexample}
\end{frame}

\begin{frame}[containsverbatim]{Υποσημειώσεις}
\begin{lstlisting}
It is easy to add a footnote \footnote{like this}.
\end{lstlisting}
\fbox{It is easy to add a footnote\footnote[frame]{like this}\\}
\end{frame}

\subsection{Floats}

\begin{frame}[containsverbatim]{Floats, Λεζάντες κλπ}
\begin{block}{Floats}
Είναι δοχεία που τοποθετούνται έξυπνα από τον μεταγλωτιστή στην θέση που τους αρμόζει στο κείμενο. Επιτρέπεται να περιέχουν λεζάντες καθώς και ετικέτες για αναφορά μέσα στο κείμενο.
\end{block}
\begin{columns}
\begin{column}{.5\textwidth}
\begin{lstlisting}
\begin{figure}[options]
\caption{Test}
\label{fig:label}
\end{figure}
\end{lstlisting}
\end{column}
\begin{column}{.5\textwidth}
\begin{lstlisting}
\begin{table}[options]
\caption{Test}
\label{tab:label}
\end{table}
\end{lstlisting}
\end{column}
\end{columns}
\end{frame}

\begin{frame}[containsverbatim]{Subfloats}
\begin{LTXexample}[basicstyle=\ttfamily\scriptsize]
\begin{figure}
\centering
  \begin{subfigure}[b]{0.3\textwidth}
  \includegraphics[width=\textwidth]{test}
  \caption{T1}
  \label{fig:test1}
  \end{subfigure}
  \quad
  \begin{subfigure}[b]{0.3\textwidth}
  \includegraphics[width=\textwidth]{test}
  \caption{T2}
  \label{fig:test2}
  \end{subfigure}
\caption{Picture of tests}
\label{fig:tests}
\end{figure}
\end{LTXexample}
\end{frame}

\subsection{Λίστες}

\begin{frame}[containsverbatim]{Απλή λίστα}
\begin{LTXexample}
\begin{itemize}
  \item A
  \item B
  \item C
  \begin{itemize}
    \item a
    \item b
  \end{itemize}
  \item D
  \item E
\end{itemize}
\end{LTXexample}
\end{frame}

\begin{frame}[containsverbatim]{Περιγραφική λίστα}
\begin{LTXexample}
\begin{description}
  \item[A:]{AA}
  \item[B:]{BB}
  \item[C:]{CC}
  \begin{description}
    \item[a:]{aa}
    \item[b:]{bb}
  \end{description}
  \item[D:]{DD}
  \item[E:]{EE}
\end{description}
\end{LTXexample}
\end{frame}

\begin{frame}[containsverbatim]{Αριθμημένη λίστα}
\begin{LTXexample}
\begin{enumerate}
  \item A
  \item B
  \item C
  \begin{enumerate}
    \item a
    \item b
  \end{enumerate}
  \item D
  \item E
\end{enumerate}
\end{LTXexample}
\end{frame}

\subsection{Πίνακες}

\begin{frame}[containsverbatim]{Στοίχιση}
\begin{LTXexample}
\begin{tabbing}
1 \= 2 \\
\> 2 \= 3\\
\> \> 3 \= 4 \\
1 \> \> 3\\
\end{tabbing}
\end{LTXexample}
\end{frame}

\begin{frame}[containsverbatim]{Πίνακες (Tabular)}
\begin{LTXexample}[pos=b,basicstyle=\ttfamily\scriptsize]
\begin{tabular}{|r|c|p{1in}|}
    \hline
    \multicolumn{3}{|c|}{\sc Title} \\
    \hline
    \hline
    \bf 1 & \bf 2 & \multicolumn{1}{c|}{\bf 3} \\
    \hline
    1 & 2 & 3333 \\
    1 & 2 & 3333 \\
    1 & 2 & 3333 \\ 
    \hline
\end{tabular}
\end{LTXexample}
\end{frame}

\begin{frame}[containsverbatim]{Πίνακες (Table)}
\begin{LTXexample}[basicstyle=\ttfamily\scriptsize]
\begin{table}
\caption{This is a table}
\label{tab:label}
\begin{center}
\begin{tabular}{|c|c|c|}
    \toprule
    \bf 1 & \bf 2 & \bf 3 \\
    \midrule
    1 & 2 & 3333 \\
    1 & 2 & 3333 \\
    1 & 2 & 3333 \\ 
    \bottomrule
\end{tabular}
\end{center}
\end{table}

Refer to Table~\ref{tab:label}
\end{LTXexample}
\end{frame}

\subsection{Γραφικά}

\begin{frame}{Υποστηριζόμενες Εικόνες}
\begin{tabular}{c p{3in}}
Μεταγλωτιστής & Είδος αρχείου \\
\hline
latex & eps\\
pdflatex & pdf, png, eps, jpg \\
xetex & pdf, png, eps, jpg \\
luatex & pdf, png, eps, jpg
\end{tabular}
\end{frame}

\begin{frame}[containsverbatim]{Γραφικά}
\begin{LTXexample}[basicstyle=\ttfamily\scriptsize]
\begin{figure}[H]
\caption{This is a figure}
\label{fig:label}
\begin{center}
\includegraphics[angle=90,width=.8\textwidth]{test}
\end{center}
\end{figure}

Refer to Fig.~\ref{fig:label}
\end{LTXexample}
\end{frame}

\subsection{Μαθηματικά}

\begin{frame}{Τύποι}
\begin{tabular}{|c|c|c|c|}
\hline
\bf{Τύπος} & \bf{Σε κείμενο} & \bf{Σε εξίσωση} & \bf{Αριθμημένα}\\
\hline
Περιβάλλον & math & displaymath & equation \\
& & equation* & \\
\hline
\LaTeX{} & \textbackslash( \ldots \textbackslash) & \textbackslash [ \ldots \textbackslash ] & \\
\hline
\TeX{} & \$ \ldots \$ & \&\& \ldots \&\& & \\
\hline
\end{tabular}
\end{frame}

\begin{frame}[containsverbatim]{Παράδειγμα 1}
\begin{LTXexample}[pos=b,basicstyle=\ttfamily\scriptsize]
This is an inline math $\forall x \in X, \quad \exists y \leq \epsilon$

but this is displayed \[\lim_{x \to \infty} \exp(-x) = 0 \]
and this is numbered in eq.~\ref{eq:label}
\begin{equation}
k_{n+1} = n^2 + \sqrt{k^2_n - k_{n-1}}
\label{eq:label}
\end{equation}
\end{LTXexample}
\end{frame}

\begin{frame}[containsverbatim]{Παράδειγμα 2}
\begin{LTXexample}[pos=b,basicstyle=\ttfamily\scriptsize]
\begin{equation}
\frac{num}{denum} = \frac{1}{\frac{denum}{num}}
\end{equation}
\[
 z = \overbrace{
   \underbrace{x}_\text{real} + 
   \underbrace{y}_\text{imaginary} i
  }^\text{complex number}
\]
\end{LTXexample}
\end{frame}

\begin{frame}[containsverbatim]{Παράδειγμα 3}
\begin{LTXexample}[pos=b,basicstyle=\ttfamily\scriptsize]
\begin{align*}
\gamma = \int_0^\infty f(x) & \mathrm{d}x \\
R & = 
\begin{bmatrix}
1 & 0 & 0 \\ 
0 & 1 & 0 \\ 
0 & 0 & \gamma 
\end{bmatrix}
\end{align*}
\end{LTXexample}
\end{frame}

\subsection{Αλγόριθμοι}

\begin{frame}[containsverbatim]{Αλγόριθμοι}
\begin{LTXexample}[basicstyle=\ttfamily\scriptsize]
\begin{algorithmic}
\Function{split}{$S$, $i$}
  \LState $S$ $\gets$ \Call{getCP}{$S[i]$}
  \LState \Return $S$
\EndFunction
\end{algorithmic}
\end{LTXexample}
\end{frame}

\begin{frame}[containsverbatim]{Πηγαίος Κώδικας 1}
\begin{LTXexample}[pos=b,basicstyle=\ttfamily\scriptsize]
\lstinputlisting[language=Python]{kinematics.py}
\end{LTXexample}
\end{frame}

\begin{frame}[containsverbatim]{Πηγαίος Κώδικας 2}
\begin{LTXexample}[pos=b,basicstyle=\ttfamily\scriptsize]
\begin{minted}[mathescape,fontsize=\scriptsize]{vhdl}
library IEEE;
use IEEE.std_logic_1164.all;
entity AND_gate is
  port(x,y: in std_logic; z: out std_logic);
end AND_gate;
\end{minted}
\end{LTXexample}

Απαιτεί Python και Pygments!
\end{frame}

\subsection{Χημεία}

\begin{frame}[containsverbatim]{Χημικές Ενώσεις 1}
\begin{LTXexample}[basicstyle=\ttfamily\scriptsize]
\chemfig{A-B}\\
\chemfig{A=B}\\
\chemfig{A~B}\\
\chemfig{A>B}\\
\chemfig{A<B}\\
\chemfig{A>:B}\\
\chemfig{A<:B}\\
\chemfig{A>|B}\\
\chemfig{A<|B}\\
\end{LTXexample}
\end{frame}

\begin{frame}[containsverbatim]{Χημικές Ενώσεις 2}
\begin{LTXexample}[pos=b,basicstyle=\ttfamily\scriptsize]
\chemfig{C(-[:0]H)(-[:90]H)(-[:180]H)(-[:270]H)} 
\quad
\chemfig{*6(=-=-=-)}
\quad
\chemfig{-(-[:45]O^{-})=[:315]O}
\end{LTXexample}
\end{frame}

\section{Πακέτα}

\begin{frame}{Πακέτα}
\begin{block}{Προοίμιο}
Το προοίμιο συνήθως είναι σταθερό. Το φτιάχνεις μία φορά και μετά χρησιμοποιείς πάντα το ίδιο. Μπορείς να το έχεις σε ένα ξεχωριστό αρχείο και να το εισάγεις με το \textbackslash input{} σε κάθε αρχείο σου.
\end{block}
\pause
\begin{alertblock}{...αλλά...}
ακόμα και έτσι, μπορεί να είναι αρκετά επίπονη και χρονοβόρα διαδικασία!
\end{alertblock}
\end{frame}

\begin{frame}[containsverbatim]{Χρήσιμα πακέτα}
\begin{lstlisting}[basicstyle=\ttfamily\scriptsize]
\usepackage{hyperref}
\usepackage{float}
\usepackage{graphicx}
\usepackage{mathtools}
\usepackage{amssymb}
\usepackage{amsthm}
\usepackage{caption}
\usepackage{subcaption}
\usepackage{listings}
\usepackage{algorithm}
\usepackage{algpseudocode}
\usepackage{tikz}
\usepackage{xcolor}
\usepackage{tabularx}
\usepackage{xltxtra}
\usepackage{xgreek}
\usepackage{xunicode}
\usepackage{minted}
\usepackage{chemfig}
\usepackage{fontspec}
\usepackage{booktabs}
\usepackage{verbatim}
\end{lstlisting}
\end{frame}

\section{Γραμματο\-σειρές}

\begin{frame}{GFS}
\url{http://www.greekfontsociety.gr/}

{\fontspec{GFS Didot}GFS Didot}

{\fontspec{GFS Neohellenic}GFS Neohellenic}

{\fontspec{GFS Artemisia}GFS Artemisia}

{\fontspec{GFS Theokritos}GFS Theokritos}
\end{frame}

\section{Βιβλιογραφία}

\begin{frame}[containsverbatim]{Ενσωματωμένη}
\begin{LTXexample}[pos=b,basicstyle=\ttfamily\scriptsize]
\begin{thebibliography}{9}
\bibitem{lamport94}
  Leslie Lamport,
  \emph{\LaTeX: a document preparation system}.
  Addison Wesley, Massachusetts,
  2nd edition,
  1994.
\end{thebibliography}
\end{LTXexample}
\end{frame}

\begin{frame}[containsverbatim]{Εξωτερική (BibTeX)}
\begin{block}{Αρχείο .bib}
Η βιβλιογραφία αποθηκεύεται εκτός του κειμένου σε ξεχωριστό αρχείο με ειδική μορφή. 
\end{block}

\begin{lstlisting}[pos=b,basicstyle=\ttfamily\scriptsize]
@book{lamport94,
  title     = {LaTeX: A document preparation system},
  author    = {Lamport, Leslie},
  year      = {1994},
  edition   = {2nd},
  publisher = {Reading, Mass: Addison-Wesley Professional. ISBN 0-201-52983-1}
}
\end{lstlisting}
\end{frame}

\begin{frame}[containsverbatim]{Αναφορές (BibTeX)}
\bf{Αυτόματη αρίθμηση και αλλαγή του στυλ πολύ εύκολα!}
\begin{lstlisting}[pos=b,basicstyle=\ttfamily\scriptsize]
\cite{lamport94} created a great document preparation system!

\bibliographystyle{plain}
\bibliography{path/to/file}
\end{lstlisting}
\end{frame}

\begin{frame}{Συμβατότητα}
\begin{itemize}
\item Mendeley
\item Zotero
\item JabRef
\item CiteULike
\item Ebib
\item Όλες τις ιστοσελίδες επιστημονικών περιοδικών (Scholar, scopus, IEEE κλπ)
\end{itemize}
\end{frame}

\section{Γραφικά}

\begin{frame}{Το TikZ}
\begin{alertblock}{PGF/TikZ}
Πακέτο που επιτρέπει να φτιάξεις γραφικά απευθείας στο \TeX{} χωρίς να εισάγεις εικόνες από άλλα προγράμματα.
\end{alertblock}
\begin{itemize}
\item Πολλά πακέτα με διαφορετικές δυνατότητες
\item Δυνατότητα μακροεντολών
\item Αλλαγή των χαρακτηριστικών εύκολα
\item Αλλαγή των χαρακτηριστικών με τέλεια ποιότητα
\end{itemize}
\end{frame}

\begin{frame}[containsverbatim]{Βασική χρήση}
\begin{LTXexample}[basicstyle=\ttfamily\scriptsize]
\begin{tikzpicture}
\draw (1,0) -- (0,0) -- (0,1) -- cycle;
\end{tikzpicture}
\end{LTXexample}
\begin{LTXexample}[basicstyle=\ttfamily\scriptsize]
\begin{tikzpicture}
\draw (0,0) .. controls (1,1) .. (2,0)
      (3,0) .. controls (3,1) and (2,1) .. (2,2);
\end{tikzpicture}
\end{LTXexample}
\end{frame}

\begin{frame}[containsverbatim]{Βασική χρήση}
\begin{LTXexample}[basicstyle=\ttfamily\scriptsize]
\begin{tikzpicture}
\foreach \x in {0,...,4} 
  \draw (\x,0) circle (0.4);
\end{tikzpicture}
\end{LTXexample}
\begin{LTXexample}[basicstyle=\ttfamily\scriptsize]
\begin{tikzpicture}
\draw[dotted]
    (0,0) node {1st node}
 -- (1,1) node {2nd node}
 -- (0,2) node {3rd node}
 -- cycle;
\end{tikzpicture}
\end{LTXexample}
\end{frame}


\begin{frame}[containsverbatim]{Προχωρημένη χρήση}
\begin{LTXexample}[basicstyle=\ttfamily\scriptsize]
\begin{tikzpicture}
  \node (S) {};
  \draw[thick,color=blue,->] (0,0) node[anchor=north west]{$\{S\}$} -- (1,0) node[anchor=south west]{$x$};
  \draw[thick,color=blue,->] (0,0) -- (0,1) node[anchor=south west]{$y$};
  \mobilebase
\end{tikzpicture}
\end{LTXexample}
\end{frame}

\end{document}
